%----------------------------------------------------------------------------------------
%    PACKAGES AND THEMES
%----------------------------------------------------------------------------------------

\documentclass[aspectratio=169,xcolor=dvipsnames]{beamer}
\usetheme{tuDortmundBeamer} % Use the custom theme created in the same directory

\usepackage{hyperref}
\usepackage{tikz}
\usepackage{siunitx}
\usepackage{graphicx} 
\usepackage{booktabs} 

\usepackage[bitstream-charter]{mathdesign}
\usepackage[T1]{fontenc} 



%----------------------------------------------------------------------------------------
%    TITLE PAGE
%----------------------------------------------------------------------------------------

\title{Your Presentation Title Here} % Title of your presentation
\subtitle{Subtitle of Your Presentation} % Subtitle, can be omitted if not needed

\author{Moritz Bosse}

\institute
{
    Department of High Energy Theory \\
    TU Dortmund University % Your institution for the title page
}
\date{\today} % Date, can be changed to a custom date

%----------------------------------------------------------------------------------------
%    PRESENTATION SLIDES
%----------------------------------------------------------------------------------------

\begin{document}

\begin{frame}[plain,noframenumbering] % Use noframenumbering to avoid numbering this slide
    % Print the title page as the first slide
    \titlepage
\end{frame}

\begin{frame}{Overview}
    % Throughout your presentation, if you choose to use \section{} and \subsection{} commands, these will automatically be printed on this slide as an overview of your presentation
    \tableofcontents
\end{frame}

%------------------------------------------------
\section{First Section}
%------------------------------------------------

\begin{frame}{Bullet Points}
    \begin{itemize}
        \item Lorem ipsum dolor sit amet, consectetur adipiscing elit
        \item Aliquam blandit faucibus nisi, sit amet dapibus enim tempus eu
        \item Nulla commodo, erat quis gravida posuere, elit lacus lobortis est, quis porttitor odio mauris at libero
        \item Nam cursus est eget velit posuere pellentesque
        \item Vestibulum faucibus velit a augue condimentum quis convallis nulla gravida
    \end{itemize}
\end{frame}

%------------------------------------------------

\begin{frame}{Blocks of Highlighted Text}
    In this slide, some important text will be \alert{highlighted} because it's important. Please, don't abuse it.

    \begin{block}{Block}
        Sample text
    \end{block}

    \begin{alertblock}{Alertblock}
        Sample text in red box
    \end{alertblock}

    \begin{examples}
        Sample text in blue box. The title of the block is ``Examples".
    \end{examples}
\end{frame}

%------------------------------------------------

\begin{frame}{Additional Block Types}
Here are some additional block types that you can use to personalize your presentation:
    \begin{blueblock}{Typical Block but in Blue}
        \begin{itemize}
          \item Just a typical block but in blue
          \item Can be used with \texttt{\textbackslash begin\{blueblock\}} environment
        \end{itemize}
    \end{blueblock}
    \begin{greenblock}{Typical Block but in Green as the Theme}
        \begin{itemize}
          \item Just a typical block but in the same green as the theme
          \item Can be used with \texttt{\textbackslash begin\{greenblock\}} environment
        \end{itemize}
    \end{greenblock}
\end{frame}

%------------------------------------------------

\begin{frame}{TU Dortmund Theme}
If you really want to use the TU Dortmund color theme, you can use the following commands to include it in your presentation:
        \begin{tublock}{Typical Block but in TU Dortmund Green}
            \begin{itemize}
              \item Just a typical block but in TU Dortmund green
              \item Can be used with \texttt{\textbackslash begin\{tublock\}} environment
            \end{itemize}
        \end{tublock}
        \begin{tuorangeblock}{Typical Block but in TU Dortmund Orange}
            \begin{itemize}
              \item Just a typical block but in TU Dortmund orange
              \item Can be used with \texttt{\textbackslash begin\{tuorangeblock\}} environment
            \end{itemize}
        \end{tuorangeblock}
\end{frame}

% ------------------------------------------------

\begingroup
\setbeamercolor{title}{bg=tugreen,fg=white}
\setbeamercolor{frametitle}{bg=tugreen,fg=white}

\begin{frame}[fragile]{Changing the TU Color Scheme}
    You can change the color scheme of title and frame title in the theme by modifying:

    \begin{itemize}
        \item \texttt{beamercolorthemeTuDortmundBeamer.sty}
    \end{itemize}

Replace the default dark green color with the TU-branded options by commenting/uncommenting:

\begin{block}{Default Dark Green (CalPolyGreen)}
    \begin{verbatim}
    \setbeamercolor{title}{bg=CalPolyGreen,fg=white}
    \setbeamercolor{frametitle}{bg=CalPolyGreen,fg=white}
    \end{verbatim}
\end{block}

\begin{block}{Bright TU Green (tugreen)}
    \begin{verbatim}
    %\setbeamercolor{title}{bg=tugreen,fg=white}
    %\setbeamercolor{frametitle}{bg=tugreen,fg=white}
    \end{verbatim}
\end{block}
\end{frame}
\endgroup

%------------------------------------------------

\begin{frame}[plain,noframenumbering]
    \begingroup
    \setbeamercolor{title}{bg=tugreen,fg=white}
    \setbeamercolor{frametitle}{bg=tugreen,fg=white}
    \titlepage
    \endgroup
\end{frame}

%------------------------------------------------

\begingroup
\setbeamercolor{title}{bg=tugreen80,fg=white}
\setbeamercolor{frametitle}{bg=tugreen80,fg=white}

\begin{frame}[fragile]{Changing the TU Color Scheme}
    If you don’t like the bright TU green, you can also use a less bright green by modifying:

    \begin{itemize}
        \item \texttt{beamercolorthemeTuDortmundBeamer.sty}
    \end{itemize}

Replace the default dark green color with TU-branded option by commenting/uncommenting:

\begin{block}{Default Dark Green (CalPolyGreen)}
    \begin{verbatim}
    \setbeamercolor{title}{bg=CalPolyGreen,fg=white}
    \setbeamercolor{frametitle}{bg=CalPolyGreen,fg=white}
    \end{verbatim}
\end{block}

\begin{block}{TU Green / 80\% (tugreen80)}
    \begin{verbatim}
    %\setbeamercolor{title}{bg=tugreen80,fg=white}
    %\setbeamercolor{frametitle}{bg=tugreen80,fg=white}
    \end{verbatim}
\end{block}
\end{frame}
\endgroup

%------------------------------------------------

\begin{frame}[plain,noframenumbering]
    \begingroup
    \setbeamercolor{title}{bg=tugreen80,fg=white}
    \setbeamercolor{frametitle}{bg=tugreen80,fg=white}
    \titlepage
    \endgroup
\end{frame}

%------------------------------------------------

\begin{frame}{Multiple Columns}
    \begin{columns}[c] % The "c" option specifies centered vertical alignment while the "t" option is used for top vertical alignment

        \column{.45\textwidth} % Left column and width
        \textbf{Heading}
        \begin{enumerate}
            \item Statement
            \item Explanation
            \item Example
        \end{enumerate}

        \column{.45\textwidth} % Right column and width
        Lorem ipsum dolor sit amet, consectetur adipiscing elit. Integer lectus nisl, ultricies in feugiat rutrum, porttitor sit amet augue. Aliquam ut tortor mauris. Sed volutpat ante purus, quis accumsan dolor.

    \end{columns}
\end{frame}

%------------------------------------------------
\section{Second Section}
%------------------------------------------------

\begin{frame}{Table}
    \begin{table}
        \begin{tabular}{l l l}
            \toprule
            \textbf{Treatments} & \textbf{Response 1} & \textbf{Response 2} \\
            \midrule
            Treatment 1         & 0.0003262           & 0.562               \\
            Treatment 2         & 0.0015681           & 0.910               \\
            Treatment 3         & 0.0009271           & 0.296               \\
            \bottomrule
        \end{tabular}
        \caption{Table caption}
    \end{table}
\end{frame}

%------------------------------------------------

\begin{frame}{Definitions and Theorems}
    \begin{definition}[Compactness]
        A topological space $X$ is \emph{compact} if every open cover of $X$ has a finite subcover.
    \end{definition}
    \begin{definition}[Paracompactness]
        A topological space $X$ is \emph{paracompact} if every open cover of $X$ admits an open locally finite refinement that also covers $X$.
    \end{definition}
    \begin{theorem}[Heine--Borel Theorem]
        A subset of $\mathbb{R}^n$ is compact if and only if it is closed and bounded.
    \end{theorem}
\end{frame}

%------------------------------------------------

\begin{frame}{Example: Heine--Borel Theorem}
    \begin{example}
        Consider the interval $[0, 1] \subset \mathbb{R}$.

        \begin{itemize}
            \item It is \textbf{closed}: it contains its endpoints 0 and 1.
            \item It is \textbf{bounded}: all points lie between 0 and 1.
            \item Therefore, by the Heine--Borel Theorem, $[0,1]$ is \textbf{compact}.
        \end{itemize}

        Now consider the open interval $(0, 1)$:
        \begin{itemize}
            \item It is \textbf{bounded} but \textbf{not closed}.
            \item It is \textbf{not compact}.
            \item For example, the open cover 
            \[
                \left\{ \left( \frac{1}{n}, 1 - \frac{1}{n} \right) \mid n \geq 2 \right\}
            \]
            covers $(0,1)$, but has \textbf{no finite subcover}.
        \end{itemize}
    \end{example}
\end{frame}

%------------------------------------------------

\begin{frame}{Figure}
    Uncomment the code on this slide to include your own image from the same directory as the template .TeX file.
    %\begin{figure}
    %\includegraphics[width=0.8\linewidth]{test}
    %\end{figure}
\end{frame}

%------------------------------------------------

\begin{frame}[fragile] % Need to use the fragile option when verbatim is used in the slide
    \frametitle{Citation}
    An example of the \verb|\cite| command to cite within the presentation:\\~

    This statement requires citation \cite{a1}.
\end{frame}

%------------------------------------------------

\begin{frame}{References}
    I generally advise against including a bibliography in presentations, but if you need to you can use the following code to include it:
    \footnotesize
    \bibliography{reference.bib}
    \bibliographystyle{apalike}
\end{frame}

%------------------------------------------------

\begin{frame}
    \Huge{\centerline{\textbf{The End}}}
\end{frame}

%----------------------------------------------------------------------------------------

\end{document}